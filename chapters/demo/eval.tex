\section{Evaluation}

  The completion of the goals and criteria described in \hyperref[analysis]{requirement analysis} should be evaluated as a measure to determine whether the system and its functionalities have accomplished its main objective according to the goal of the research project. 

  To evaluate whether the system has accomplished the goal of the research project properly, the implementation of each user stories described in \autoref{user_stories} will be listed in the table below: 
  
  \begin{tabularx}{\linewidth}{p{0.4\textwidth} p{0.6\textwidth}}
   \caption{Implementations of the primary use cases of the system} \\
    \toprule
    User Story & Implementation \\
    \midrule
    As a stakeholder, I want to verify users, so that the company can have more confidence that the existing user base is trustworthy & Providing an HTTP endpoint to trigger a validation process based on its request body and returning a validation result that contains a numerical value to determine the probability of the customer being a fraud \\

    \hline

    As an employee, I want to be notified when a user seems suspicious, so that I can do necessary actions accordingly & Providing an AMQP message broker and publishing the message to an AMQP exchange whenever a validation process is completed. An additional AMQP message consumer can be implemented to send the necessary notification \\

    \hline 

    As an employee, I want to manage my own rule to validate users, so that I can use my expertise to find suspicious customers as efficiently as possible without the communication overhead with other teams & Providing endpoints to create, read, delete and update validation rules. Additionally, endpoints to create and delete runtime secrets to store confidential information (such as API keys) are also implemented \\
    \bottomrule
  \end{tabularx}

  Furthermore, a further evaluation on the criteria listed in \autoref{criteria} should also be done, to further evaluate the quality of the system created as the result of this research project. 
  
  \begin{tabularx}{\linewidth}{p{0.3\textwidth} p{0.2\textwidth} p{0.5\textwidth}}
   \caption{Completion Table of the Systems and Software Quality Standard (ISO 25010)} \\
    \toprule
    & Importance & Completion \\
    \midrule

    \multicolumn{3}{@{}l}{\textbf{Functional stability}}\\
    Completeness & Very important & Features listed on the user stories defined in \autoref{user_stories} are implemented \\
    Correctness & Very important & Thorough tests on the \verb;ValidationEngine; class and its components \\
    Appropriateness & Important & Provided an overview of a validation process, and the \verb;ValidationResult; model can be further extended to contain additional information \\

    \multicolumn{3}{@{}l}{\textbf{Reliability}}\\
    Maturity & Important & The system handles exception well and doesn't break when any error occurs \\
    Availability & Very important & The system is runnable via a single command, and a live environment is also available \\
    Fault tolerance & Important & The system can still run without the RabbitMQ instance and Redis by providing fallback options (e.g. \verb;InMemoryStore; as a data store if Redis is not available) \\
    Recoverability & Very important & Only if Redis is available, otherwise not complete \\

    \multicolumn{3}{@{}l}{\textbf{Performance efficiency}}\\
    Time behavior & Important & Running a validation process as an asynchronous operation to prevent long response time \\
    Resource Utilization & Not important & Not complete \\
    Capacity & Not important & Not complete \\

    \multicolumn{3}{@{}l}{\textbf{Usability}}\\
    Appropriateness Recognizability & Not important & Not complete \\
    Learnability & Not important & Not complete \\
    Operability & Very important & Provided a UI as an interface to interact with the FDS and \verb;ValidationEngine; \\
    User Error Protection & Somewhat important & Provided autocomplete input on the UI to prevent invalid JSONPath expression \\
    User Interface Aesthetics & Very important & Provided a simple UI and a consistent design across all views \\
    Accessibility & Not important & Not complete \\

    \multicolumn{3}{@{}l}{\textbf{Security}}\\
    Confidentiality & Somewhat important & Not complete \\
    Integrity & Somewhat important & Not complete \\
    Non-repudiation & Somewhat important & Not complete \\
    Accountability & Not important & Not complete \\
    Authenticity & Not important & Not complete \\

    \multicolumn{3}{@{}l}{\textbf{Compatibility}}\\
    Co-existence & Somewhat important & All components of the system works well with docker compose \\
    Interoperability & Very important & The FDS shares data to the remaining components of the system (UI displays data provided by the FDS and RabbitMQ forwards the message published by the FDS) \\

    \multicolumn{3}{@{}l}{\textbf{Maintainability}}\\
    Modularity & Important & Thorough testing on the public interface of a component and not its implementation \\
    Reusability & Not important & Created several utility functions to be used in different components \\ 
    Analysability & Somewhat important & Not complete \\
    Modifiability & Very important & Thorough integration test to make sure any modification won't break the specification of the system as a whole \\
    Testability & Very important & Write unit tests before implementing the feature and automatically run the tests on every new merge request \\

    \multicolumn{3}{@{}l}{\textbf{Portability}}\\
    Adaptability &  Not important & Used Docker, so that the application can be run on every operating system \\
    Installability & Important & Implemented necessary steps in the Dockerfile for an easier build and install step \\
    Replaceability & Not important & Not complete \\

    \bottomrule
  \end{tabularx}