\chapter{Introduction}

Fraud is an activity where someone intentionally deceives another person / system for any unlawful gain. The need to prevent as many fraud activities as possible should be one of the main priorities for businesses, as the number of fraud cases increase every year. In year 2021, the US Federal Trade Commission (FTC) received 2.8 million fraud reports, 70\% more in comparison to the fraud reports in 2020 \autocite{ftc}. Many businesses might already have some experience in handling such fraud cases, but an automated system that could detect and possibly prevent fraud activity with minimal supervision would be beneficial to reduce future risks while providing the possibility and capacity to scale their product.

Fraud detection system is a system or program that uses a set of processes or techniques to detect fraudulent activities based on the input data in an automated way. A fraud detection system could also possibly prevent further fraud activities by running a certain action (e.g. Blocking a fraudulent customer). 

As a business scale, it is often a good idea to split the responsibility of a certain domain to its own team, consisting of several people that focus solely on the given area. Large businesses are often built on top of multiple teams, working together as a whole, but usually handle their own responsibilities, have their own goals and use a different technology stack and practices. Given the architecture principles and the autonomy of the teams, how could fraud detection and prevention centrally be managed?

\section{Background and Motivation}
The background and motivation of the thesis is to get my bachelor degree.
\section{Goal}
Goal of the thesis is to build a cool software
\section{Scope}

  The main purpose of this research project is to explore any possibility to facilitate a collaboration for several autonomous teams in a single fraud detection process while also providing crucial functionalities such as a notification system. 

  This project won't necessarily undertake aspects such as GDPR compliance and authentication of the system. Therefore, the system is not ready for production and further improvements on these aspects is required. The result of this research project is an explorative work, and may be used as a base for future projects in similar domain.  