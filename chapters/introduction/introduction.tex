\chapter{Introduction}

Fraud is an activity where someone intentionally deceives another person / system for any unlawful gain. The need to prevent as many fraud activities as possible should be one of the main priorities for businesses, as the number of fraud cases increase every year. In year 2021, the US Federal Trade Commission (FTC) received 2.8 million fraud reports, 70\% more in comparison to the fraud reports in 2020 \autocite{ftc}. Many businesses might already have some experience in handling such fraud cases, but an automated system that could detect and possibly prevent fraud activity with minimal supervision would be beneficial to reduce future risks while providing the possibility and capacity to scale their product.

Fraud detection system is a system or program that uses a set of processes or techniques to detect fraudulent activities based on the input data in an automated way. A fraud detection system could also possibly prevent further fraud activities by running a certain action (e.g. Blocking a fraudulent customer). 

As a business scale, it is often a good idea to split the responsibility of a certain domain to its own team, consisting of several people that focus solely on the given area. Large businesses are often built on top of multiple teams, working together as a whole, but usually handle their own responsibilities, have their own goals and use a different technology stack and practices. Given the architecture principles and the autonomy of the teams, how could fraud detection and prevention centrally be managed?

\section{Background and Motivation}

  With multiple teams working simultaneously, it is possible that some security checks already exist. Having this in mind, a common ground to integrate existing and a possibility to implement new security checks should be established. Unfortunately, every team has their own agenda and priorities, making it almost impossible to build a unified system that scales without having a huge inter-team dependency. 

  A solution would be that every team implements some fraud detection in their systems and provides an interface for other services to access the data to be used by other services. As part of the solution proposed, a centralized system is needed to act as a gateway that is easily accessible. The centralized system can provide a possibility to build a pipeline of security checks, enabling the user to create a custom flow of checks out of the existing services. Grouping all the check results in a single list would be convenient for the user in interpreting the check results from various checks across multiple teams as a whole. The interface of the service itself should be usable by anyone without any technical background. Hence, making it even easier to manage fraud activity.
\section{Goal and Scope}

  The goal of this research project is to explore the possibility to build a system to detect and/or prevent fraud activities while providing the opportunity for multiple autonomous teams to contribute to the process by leveraging their domain knowledge and expertise.

  This project won't necessarily undertake aspects such as GDPR compliance and authentication process. Therefore, the system is not ready for production and further improvements on these aspects are required. The result of this research project is an explorative work, and may be used as a base for future projects in similar domain.  
\section{Scope}
Scope of the thesis is to not build a new internet protocol