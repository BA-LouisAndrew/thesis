\section{Background and Motivation}

  With multiple teams working simultaneously, it is possible that some security checks already exist. Having this in mind, a common ground to integrate existing and a possibility to implement new security checks should be established. Unfortunately, every team has their own agenda and priorities, making it almost impossible to build a unified system that scales without having a huge inter-team dependency. 

  A solution would be that every team implements some fraud detection in their systems and provides an interface to access the data to be used by other services. As part of the solution proposed, a centralized system is needed to act as a \emph{gateway} that is easily accessible. The centralized system can provide a possibility to build a pipeline of security checks, enabling the user to create a custom flow of checks out of the existing services. Grouping all the check results in a single list would be convenient for the user in interpreting the check results from various checks across multiple teams as a whole. The interface of the system itself should be usable by anyone without any technical background. Hence, making it even easier to manage fraud activity.