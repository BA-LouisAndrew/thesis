\chapter{Conclusion}

  In this chapter, the results and findings from previous chapters are summarized and an outlook on future improvements of the system is discussed.

  \section{Summary}

    Fraud detection and prevention plays a crucial role in building a profitable business by reducing the risk of damage done by any fraudulent activity. Furthermore, for organizations that utilize the autonomous teams structure, establishing a robust system for collaboration is important, so that the teams can work collectively on the primary goal of the organization.

    This research project aims to introduce a system that enables independent teams to collaborate in a fraud detection process, by creating validation rules as their contribution. Each team could create validation rules to validate certain characteristics of a customer, based on their domain knowledge and views on how a fraudulent customer might be. All validation rules created by different teams will be evaluated by the system and as the result, the system outputs a probability of the corresponding customer being a fraud, by combining the result of every validation rule evaluation into a single numerical value, ranging from zero to one. 
    
    The system publishes each validation result into an exchange of a messaging system, that will broadcast the message into every message consumer bound to the exchange. Additional independent systems can be implemented as a message consumer to consume the message from the exchange and run actions accordingly. A UI is also provided as a  graphical interface to manage validation rules and display the progress of a validation process in real time. 
  
  \section{Outlook}

    The system built in this research project is far from perfect. Systems are living products that can and should be improved over time. One of the possible major improvement of the system is the application of any authentication on the system. By implementing some sort of authentication mechanism, the security aspect of the system can be hugely improved, and any unauthorized use can be prevented. 

    Furthermore, it might be sometimes useful to group the validation rules, and run the rule evaluations only for certain groups. By assigning each rule into a group, there can be multiple types of validation processes, which can be modified based on the needs and requirements of the particular use case. Last but not least, a solution to make sure that the data of the customer is safe and compliant with the current GDPR rules is definitely needed. 