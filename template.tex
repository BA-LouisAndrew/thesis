
%----------------------------------------------------------------------------------------
% Bachelorthesis. Andrew, Louis.
% Template by: Prof. C. Schmidt 	 
%----------------------------------------------------------------------------------------
\documentclass[oneside,bibliography=totocnumbered,BCOR=5mm]{scrbook}% Voreinstellungen entfernt.

\usepackage[latin1]{inputenc}
\usepackage{amsmath, amsthm, amssymb}
\usepackage[english]{babel} % Language hyphenation and typographical rules
\usepackage{marvosym}
\usepackage{graphics}
\usepackage{csquotes}
\newtheorem{satz}{Satz}[chapter]
\theoremstyle{definition} 
\newtheorem{definition}[satz]{Definition} 
\theoremstyle{definition} 
\newtheorem{lemma}[satz]{Lemma} 
\theoremstyle{definition} 
\newtheorem{bemerkung}[satz]{Bemerkung}
\theoremstyle{definition} 
\newtheorem{korollar}[satz]{Korollar} 
\theoremstyle{definition}
\newtheorem{beispiel}[satz]{Beispiel} 
\theoremstyle{definition} 
\newtheorem{algorithmus}{Algorithmus} 
\newenvironment{beweis}{\begin{proof}[Beweis]}{\end{proof}}
\usepackage[hyphens]{url}
\usepackage{hyperref}

\usepackage[backend=bibtex, style=numeric]{biblatex}
\addbibresource{template.bib}

\usepackage{blindtext} % Package to generate dummy text throughout this template 

\usepackage[sc]{mathpazo} % Use the Palatino font
\usepackage[T1]{fontenc} % Use 8-bit encoding that has 256 glyphs
\linespread{1.05} % Line spacing - Palatino needs more space between lines
\usepackage{microtype} % Slightly tweak font spacing for aesthetics

\usepackage[hmarginratio=1:1,top=32mm,columnsep=20pt]{geometry} % Document margins
\usepackage[hang, small,labelfont=bf,up,textfont=it,up]{caption} % Custom captions under/above floats in tables or figures
\usepackage{booktabs} % Horizontal rules in tables
\usepackage{lettrine} % The lettrine is the first enlarged letter at the beginning of the text
\usepackage{enumitem} % Customized lists
\setlist[itemize]{noitemsep} % Make itemize lists more compact

\usepackage{titling} 

\usepackage{listings}
\usepackage{color}

\usepackage{xcolor}

\usepackage{graphicx} % Readjust height and width of images
\usepackage{float} % Makes the figure float?!
\usepackage{hyperref}

\definecolor{lightgray}{rgb}{.9,.9,.9}
\definecolor{darkgray}{rgb}{.4,.4,.4}
\definecolor{purple}{rgb}{0.65, 0.12, 0.82}

\definecolor{eclipseStrings}{RGB}{42,0.0,255}
\definecolor{eclipseKeywords}{RGB}{127,0,85}
\colorlet{numb}{magenta!60!black}

\lstdefinelanguage{json}{
    basicstyle=\normalfont\ttfamily,
    commentstyle=\color{eclipseStrings}, % style of comment
    stringstyle=\color{eclipseKeywords}, % style of strings
    numbers=left,
    numberstyle=\scriptsize,
    stepnumber=1,
    numbersep=8pt,
    showstringspaces=false,
    breaklines=true,
    frame=lines,
    backgroundcolor=\color{lightgray}, %only if you like
    string=[s]{"}{"},
    comment=[l]{:\ "},
    morecomment=[l]{:"},
    literate=
        *{0}{{{\color{numb}0}}}{1}
         {1}{{{\color{numb}1}}}{1}
         {2}{{{\color{numb}2}}}{1}
         {3}{{{\color{numb}3}}}{1}
         {4}{{{\color{numb}4}}}{1}
         {5}{{{\color{numb}5}}}{1}
         {6}{{{\color{numb}6}}}{1}
         {7}{{{\color{numb}7}}}{1}
         {8}{{{\color{numb}8}}}{1}
         {9}{{{\color{numb}9}}}{1}
}

\lstdefinelanguage{js}{
  keywords={typeof, new, true, false, catch, function, return, null, catch, switch, const, if, in, while, do, else, case, break, await},
  keywordstyle=\color{blue}\bfseries,
  ndkeywords={class, export, boolean, throw, implements, import, this, type, interface, async},
  ndkeywordstyle=\color{darkgray}\bfseries,
  identifierstyle=\color{black},
  sensitive=false,
  comment=[l]{//},
  morecomment=[s]{/*}{*/},
  commentstyle=\color{purple}\ttfamily,
  stringstyle=\color{red}\ttfamily,
  morestring=[b]',
  morestring=[b]"
}

\lstdefinelanguage{docker}{
  keywords={FROM, RUN, COPY, ADD, ENTRYPOINT, CMD,  ENV, ARG, WORKDIR, EXPOSE, LABEL, USER, VOLUME, STOPSIGNAL, ONBUILD, MAINTAINER},
  keywordstyle=\color{blue}\bfseries,
  identifierstyle=\color{black},
  sensitive=false,
  comment=[l]{\#},
  commentstyle=\color{purple}\ttfamily,
  stringstyle=\color{red}\ttfamily,
  morestring=[b]',
  morestring=[b]"
}

\lstset{
   language=js,
   backgroundcolor=\color{lightgray},
   extendedchars=true,
   basicstyle=\normalfont\ttfamily,
   showstringspaces=false,
   showspaces=false,
   numbers=left,
   numberstyle=\footnotesize,
   numbersep=9pt,
   tabsize=2,
   breaklines=true,
   showtabs=false,
   captionpos=b,
   xleftmargin=0.5cm,
   aboveskip=0.5cm
}

\begin{document}

% Titelseite
% \pagestyle{empty}       % keine Seitennummer
\begin{titlepage}
\begin{center}
\includegraphics{images/HTW_Berlin_Logo_farbig.jpg}
\linebreak[4]
\linebreak[4]
\linebreak[4]
\linebreak[4]
\textit{\large Design \& Implementation of a Fraud Detection System for Autonomous Teams (Total pages should be: 50 [without Attachments])}
\linebreak[4]
\linebreak[4]
\linebreak[4]
Abschlussarbeit 
\linebreak[4]
\linebreak[4]
zur Erlangung des akademischen Grades: 
\linebreak[4]
\linebreak[4]
\textbf{Bachelor of Science (B.Sc.)} 
\linebreak[4]
\linebreak[4]
an der
\linebreak[4]
\linebreak[4]
Hochschule f\"ur Technik und Wirtschaft (HTW) Berlin
\linebreak[4]
Fachbereich 4: Informatik, Kommunikation und Wirtschaft
\linebreak[4]
Studiengang \textit{Angewandte Informatik}
\linebreak[4]
\linebreak[4]
\linebreak[4]
1. Gutachterin: Prof. Dr. Christin Schmidt\linebreak[4]
2. Gutachter: MSc. Tobias Dumke\linebreak[4]
\linebreak[4]
\linebreak[4]
\linebreak[4]
\linebreak[4]
Eingereicht von Louis Andrew [s0570624]
\linebreak[4]
\linebreak[4]
\linebreak[4]
\linebreak[4]
Datum
\linebreak[4]
\input{others/version}

\end{center}
\end{titlepage}

\newpage
\thispagestyle{empty}       % keine Seitennummer

\section*{Abstract}

% motivation goals, features architecture, summary, evaluation

The necessity of detecting and preventing fraud activities should be one of the main focus for businesses, to minimize the damage done through the unlawful activity while trying to increase their profit. Nowadays, businesses are usually built upon multiple teams that work independently within different domain and specialties. This paper explores the possibility to build a fraud detection system that enables collaboration between multiple autonomous teams in a single fraud detection process. 

A requirement analysis is done to properly define the requirements and features of the system built as the proposed solution. The system enables the teams to incorporate their own security check as one of the validation rule, based on their view and understanding on how a fraudulent customer might be. Each team can manage their own validation rule independently, without the communication overhead to other teams using the graphical user interface, provided as part of the system. During a validation process, each validation rule will be evaluated, and each rule evaluation result affects the resulting fraud score to determine the probability of the customer being a fraud. The result of a validation process will then be published to an AMQP exchange, where an additional message consumer can be implemented to process the data when needed.

The quality standards listed on ISO/IEC 25010 is used as a metric to evaluate the system built as the result of this research project. 

\newpage
\thispagestyle{empty}
\section*{Zusammenfassung}

\begin{otherlanguage}{ngerman}
Betrugsaktivit\"aten aufzudecken und zu verhindern, sollte einer der Schwerpunkte f\"ur Unternehmen sein, um den durch unrechtm\"a{\ss}ige Aktivit\"aten verursachten Schaden zu minimieren, w\"ahrend sie versuchen, ihren Gewinn zu erh\"ohen. Heutzutage sind Unternehmen in der Regel aus mehreren Teams aufgebaut, die unabh\"angig voneinander in verschiedenen Bereichen und Spezialgebieten arbeiten. In diesem Forschungsprojekt wird die M\"oglichkeit untersucht, ein Betrugserkennungssystem zu entwickeln, das die Zusammenarbeit zwischen mehreren autonomen Teams in einem einzigen Betrugserkennungsprozess erm\"oglicht. 

Eine Anforderungsanalyse wird durchgef\"uhrt, um die Anforderungen und Funktionen des Systems, das als vorgeschlagene L\"osung entwickelt wird, genau zu definieren. Das System erm\"oglicht den Teams, ihre eigene Sicherheits\"uberpr\"ufung als eine der Validierungsregeln einzubauen, basierend auf ihrer Sicht und ihrem Verst\"andnis, wie ein betr\"ugerischer Kunde sein k\"onnte. Jedes Team kann seine eigene Validierungsregel unabh\"angig und ohne Kommunikationsaufwand mit anderen Teams verwalten, indem es die im System enthaltene grafische Benutzeroberfl\"ache nutzt. W\"ahrend eines Validierungsprozesses wird jede Validierungsregel bewertet, und jedes Ergebnis der Regelbewertung wirkt sich auf die resultierende Betrugsbewertung aus, um die Wahrscheinlichkeit zu bestimmen, dass es sich bei dem Kunden um einen Betr\"uger handelt. Das Ergebnis eines Validierungsprozesses wird dann in einem \emph{AMQP-Exchange} ver\"offentlicht, wo ein zus\"atzlicher \emph{message consumer} implementiert werden kann, um die Daten bei Bedarf zu verarbeiten.

Die in ISO/IEC 25010 aufgef\"uhrten Qualit\"atsstandards werden als Ma{\ss}stab f\"ur die Bewertung des im Rahmen dieses Forschungsprojekts entwickelten Systems verwendet. 
\end{otherlanguage}

\clearpage
\pagenumbering{roman}% Seitennummerierung "roemisch"

\tableofcontents  

 \listoffigures

 \listoftables

 \lstlistoflistings

\newpage

\pagenumbering{arabic}  
 
\chapter{Introduction}

Fraud is an activity where someone intentionally deceives another person / system for any unlawful gain. The need to prevent as many fraud activities as possible should be one of the main priorities for businesses, as the number of fraud cases increase every year. In year 2021, the US Federal Trade Commission (FTC) received 2.8 million fraud reports, 70\% more in comparison to the fraud reports in 2020 \autocite{ftc}. Many businesses might already have some experience in handling such fraud cases, but an automated system that could detect and possibly prevent fraud activity with minimal supervision would be beneficial to reduce future risks while providing the possibility and capacity to scale their product.

Fraud detection system is a system or program that uses a set of processes or techniques to detect fraudulent activities based on the input data in an automated way. A fraud detection system could also possibly prevent further fraud activities by running a certain action (e.g. Blocking a fraudulent customer). 

As a business scale, it is often a good idea to split the responsibility of a certain domain to its own team, consisting of several people that focus solely on the given area. Large businesses are often built on top of multiple teams, working together as a whole, but usually handle their own responsibilities, have their own goals and use a different technology stack and practices. Given the architecture principles and the autonomy of the teams, how could fraud detection and prevention centrally be managed?

\section{Background and Motivation}

  With multiple teams working simultaneously, it is possible that some security checks already exist. Having this in mind, a common ground to integrate existing and a possibility to implement new security checks should be established. Unfortunately, every team has their own agenda and priorities, making it almost impossible to build a unified system that scales without having a huge inter-team dependency. 

  A solution would be that every team implements some fraud detection in their systems and provides an interface for other services to access the data to be used by other services. As part of the solution proposed, a centralized system is needed to act as a gateway that is easily accessible. The centralized system can provide a possibility to build a pipeline of security checks, enabling the user to create a custom flow of checks out of the existing services. Grouping all the check results in a single list would be convenient for the user in interpreting the check results from various checks across multiple teams as a whole. The interface of the service itself should be usable by anyone without any technical background. Hence, making it even easier to manage fraud activity.
\section{Goal and Scope}

  The goal of this research project is to explore the possibility to build a system to detect and/or prevent fraud activities while providing the opportunity for multiple autonomous teams to contribute to the process by leveraging their domain knowledge and expertise.

  This project won't necessarily undertake aspects such as GDPR compliance and authentication process. Therefore, the system is not ready for production and further improvements on these aspects are required. The result of this research project is an explorative work, and may be used as a base for future projects in similar domain.  
\section{Scope}
Scope of the thesis is to not build a new internet protocol
\chapter{Fundamentals}

  This chapter describes the context of the research project and acts as a bridge to provide the reader a better understanding on necessary aspects before moving forward to the requirement analysis. 

  \section{Fraud Detection System}

    Fraudulent activities have always been a problem for businesses and can even be traced back to the year 300 B.C., when the earliest attempt of a fraud activity is recorded. 

    By learning from previous mistakes and with the help of the rapid progress of technologies, fraud detection techniques are developed to prevent further damage done by this malicious attempt. 
    
    A fraud detection system is a system that incorporates one or more fraud detection technique to detect any fraudulent entity. A fraud detection system works by accepting an input data and returning a sort of identifier that determines whether the entity is fraudulent. Usually, the output is a numerical value that represents the probability of the entity being a fraud rather than a plain true or false value. 

    \subsection{Statistical Fraud Detection Methods}
    
      Nowadays, statistical fraud detection methods are widely used to detect fraudulent entities. There are two types of statistical fraud detection; \emph{supervised} and \emph{unsupervised}. According to \autocite{statistical-fds}, a supervised fraud detection method works by training a model to make a clear distinction between a fraudulent and non-fraudulent entity. In comparison, an unsupervised fraud detection method identifies a fraudulent entity by specifying an unusual behavior or attribute of the certain entity.

    \subsection{Rule-based Approach}

      Rule-based approach fraud detection technique is an unsupervised fraud detection method that evaluates a certain entity against a pre-defined list of rules. Bolton and David mentioned in \autocite{statistical-fds}, that an unsupervised method is useful to collect data of how a fraudulent entity might be when there's not much prior knowledge.

      Unfortunately, Kou et al. argued in \autocite{1297040}, that rule-based approach can be difficult to manage, as the rules require a time-consuming configuration for each fraud possibility as well as the adaptation of the rules itself. 

  \section{Hypertext Transfer Protocol (HTTP)}
  
    According to the official specification, HTTP is an \emph{"an application-level protocol for distributed, collaborative, hypermedia information systems"} \autocite{http-rfc}. HTTP is often used to transfer multimedia data between a client and a server in a client-server architecture. 
    
    \subsection{HTTP Request}

      An HTTP request is a request message, sent by a client to the server, containing information such as the HTTP method to be applied, identifier of the resource as well as the HTTP version to be used \autocite[\enquote{5 Request}]{http-rfc}. The main purpose of an HTTP request is to apply a particular method\footnote{The HTTP methods available for an HTTP request can be found in \autocite[\enquote{5.1.1 Method}]{http-rfc}.} on the resource located on the server. 
      
      An HTTP request usually contains an absolute request-URI, which specifies the URI of the resource, on which the request should be applied.  

    \subsection{HTTP Response}

      An HTTP response is the response from the server as the interpretation result of a particular HTTP request \autocite[6 Response]{http-rfc}. An HTTP response contains a status-line, which includes a status code and status phrase as an identifier on how the server interprets the HTTP request\footnote{The complete definition on each status-code and its meaning can be found in \autocite[\enquote{10 Status Code Definitions}]{http-rfc}.}. 

    \subsection{Header Fields}

      Header fields are additional information passed in an HTTP message \autocite[\enquote{4.2 Message Headers}]{http-rfc}. In an HTTP request message, request headers are additional information sent by the client to the server and acts as a request modifier \autocite[\enquote{5.3 Request Header Fields}]{http-rfc}. In an HTTP response message, response headers are additional information from the client to the server, containing information regarding the response out of the corresponding status line \autocite[\enquote{6.2 Response Header Fields}]{http-rfc}.

    \subsection{Body}
    
      A message body is used to transfer an entity body of an HTTP message, and it contains the data transferred by the client (HTTP request message body) or by the server (HTTP response message body) \autocite[\enquote{Message Body}]{http-rfc}. The message body is optional, it can only be used if the method specification of the particular request allows it. 


\chapter{Anforderungserhebung und -analyse}
[Beschreibung der Erhebung, Granularisierung und Priorisierung der zu Grunde liegenden Anforderungen]
\section{Nutzer- und Systemanforderungen}
\subsection{Funktionale Anforderungen}
\subsubsection{Obligatorisch (MUSS)}
\subsubsection{Fakultativ (Kann)}
\subsection{Nicht-funktionale Anforderungen}

\subsubsection{Obligatorisch (MUSS)}
\subsubsection{Fakultativ (Kann)}
\section{...}


\chapter{Konzeption \& Entwurf}
[Beschreibung des Entwurfs auf Basis der Methodologie / der geplanten Vorgehensweise zur Probleml\"osung im Kontext der Anforderungen (i.A. der Art der Arbeit)]
\section{Prozess}
\section{Systemarchitektur}
\section{Softwarearchitektur}
\section{Schnittstellen}
\section{Datenmanagement}
\section{...}
\chapter{Implementation}

As the design of the system is defined and the functionalities of each component of the system is clear, the implementation of the system can finally start. This chapter describes the detailed information on the implementation of the structure and functionalities listed in \autoref{chapter:concept}.


\section{Technologies and Architecture}

Before implementing a specific functionality of the system, each component of the system needs to be setup and configured, so that an iterative development process can be done correctly. The prerequisites that need to be met before setting up the projects are: 

\begin{itemize}
 \item Node.JS version 16 is installed
 \item NPM is installed
 \item Docker is installed
 \item Git is installed
\end{itemize}

For certain components of the system, a live environment is configured, so that the system is accessible via a URL, without having to set it up on a local machine. 


 \subsection{User Interface}
 As mentioned in \autoref{concept_ui}, the UI is a web application, built using Vue3 and TypeScript.

  \subsubsection{Setup}
   Nowadays, most front end projects use some kind of build tool to help build, test, and develop web applications\footnote{For detailed information on the role of a build tool in front end development, please take a look at \autocite{Odell2014}.}. The build tool \emph{Vite}\footnote{\emph{Vite} is an open source front end build tool. GitHub repository: \url{https://github.com/vitejs/vite}.} is used to build, test and develop the UI. A new Vite project is created by running the following command in a shell terminal:
  
  \begin{lstlisting}[caption={Creating a new Vite project (Shell)}]
 npm create vite@latest ui --template vue-ts
  \end{lstlisting}

  A version control to track and manage the progress done in this particular project is also used. The version control used in this project is git and a code hosting platform for version control is also used for the project (GitHub).

  \subsubsection{Component Library}
  A component library is used in this project to speed up the development. The component library used in this project is \emph{NaiveUI}\footnote{\emph{NaiveUI} is a Vue3 component library. GitHub repository: \url{https://github.com/TuSimple/naive-ui}.}. NaiveUI is chosen because it contains several components that are suitable for the project (for example: \textsc{Timeline}, \textsc{Menu}) and it also supports TypeScript out of the box.

  \subsubsection{Code Style}
  To enforce a consistent code style and comply to the best practices of a TypeScript project, a code formatter (\emph{Prettier}) and linter (\emph{ESLint}) are used in this project. 

  \subsubsection{Docker}
  \textbf{TODO. Implementation not done}

  \subsubsection{CI/CD}
  A CI/CD process is configured within the project, to run certain actions on specific events that happen in the codebase. The tool used to enable this functionality is \emph{GitHub actions}\footnote{\emph{GitHub actions} is a CI/CD platform, created by GitHub and can be used in all repositories hosted on GitHub. Homepage: \url{https://github.com/features/actions}.}. The configured CI/CD actions in this project are:

   \begin{itemize}
    \item Run test and build, when there's a new pull request (PR)\footnote{\emph{Pull request (PR)} is a request from a developer to merge certain changes on a dedicated branch to the main branch of a repository.} to \emph{main} branch
    \item Run release and bump version of the app, when there's a new commit to \emph{main} branch
   \end{itemize}
  
  \subsubsection{Deployment}
  A live environment is available for the UI. The live environment is made possible by using \emph{Netlify}\footnote{\emph{Netlify} is a hosting platform with a git-based workflow. Homepage: \url{https://www.netlify.com/}.}. Every change made to the \emph{main} branch will be built and deployed on the Netlify platform automatically. 

 \subsection{FDS}
 As described in \autoref{concept_fds}, the FDS is a server side application, built with Node.JS and TypeScript. To build a REST API with ease, the Express.JS framework will also be used in this project. 

  \subsubsection{Setup}
  To set up a new Node.JS project, run the following command in a shell terminal:

   \begin{lstlisting}[caption={Creating a new Node.JS program (Shell)}]
 mkdir fds
 cd ./fds
 npm init -y
   \end{lstlisting}
   
  To use the TypeScript language rather than plain JavaScript, the TypeScript compiler needs to be installed and used in this project. The TypeScript compiler can also be configured to be more suitable to the personal preferences of the developer as well as the requirements of third party libraries used by the project. To install TypeScript as a development dependency\footnote{In a Node.JS project, development dependency is a third party library used only for development purposes.} and to initiate the configuration file of the TypeScript compiler, run the following commands in a shell terminal:
  
   \begin{lstlisting}[caption={Installing and configuring TypeScript compiler (Shell)}]
 npm i -D typescript
 tsc --init
   \end{lstlisting}
  
  To install Express.JS in the current project, run the following command in a shell terminal:
  
   \begin{lstlisting}[caption={Installing Express.JS (Shell)}]
 npm i express
   \end{lstlisting}
  
  \subsubsection{Database Connection with ORM (Prisma)}
  \textbf{TODO}
  
  \subsubsection{Tsoa and Swagger}
  Even though Express.JS provides a declarative and easy way to build a server side application, it's built with JavaScript in mind. Therefore, even though TypeScript can be used with Express.JS, it doesn't use all the potential functionalities that TypeScript provides. \emph{Tsoa}\footnote{\emph{Tsoa} GitHub repository: \url{https://github.com/lukeautry/tsoa}.} is a framework with integrated openAPI compiler to build server side applications that can leverage TypeScript to its potential. \emph{Tsoa} helps an express application to have the following functionalities out of the box:

   \begin{itemize}
    \item Generate Swagger\footnote{\emph{Swagger} is a tool to document server side APIs. Homepage: \url{https://swagger.io/}.} specification based on HTTP controller code
    \item Generate Swagger schema based on TypeScript interfaces
    \item Generate Swagger schema descriptions based on \emph{jsDoc}\footnote{\emph{jsDoc} is a tool to generate API documentation, similar to \emph{JavaDoc}. Homepage: \url{https://jsdoc.app/}.} comments on the source code
   \end{itemize}
  
  Other than that, \emph{tsoa} provides an alternative syntax to build an Express.JS application in a more object-oriented way. \emph{Tsoa} works by compiling the code, with the help of the TypeScript compiler into a regular Express.JS application, built with JavaScript. 

  \subsubsection{Code Style}
  Identical to the UI project, a code formatter (\emph{Prettier}) and linter (\emph{ESLint}) are also used in this project. The configuration for the code formatter and linter is slightly different in comparison to the UI project, as certain code style rules doesn't apply to a server side application\footnote{In the UI project, there are certain code style rules regarding HTML elements.}.

  \subsubsection{Docker}
  The application will be built and run as a Docker container. To be able to run an application as a Docker container, a \emph{Dockerfile} is needed as a list commands needed to be run to assemble the particular image. The commands used to assemble a Docker container for the FDS are:

   \begin{lstlisting}[caption={Dockerfile for FDS (Docker)}, language=docker]
 FROM node:16-alpine

 ARG ARG_1 # Arguments passed by --build-args flag
 ENV ARG_1=$ARG_1 # Environment variable of the image

 WORKDIR /app
 COPY ["package.json", "package-lock.json*", "./"]

 RUN npm ci # Install packages
 COPY . .
 # Additional steps to setup the application

 RUN npm run build
 ENV NODE_ENV=production

 CMD [ "node", "./dist/src/main.js" ]
   \end{lstlisting}
  
  \subsubsection{CI/CD}
  A CI/CD process is also configured for this project using GitHub actions. The actions configured in this project are:

   \begin{itemize}
    \item Run test and build, when there's a new PR to the \emph{main} branch
    \item Run release, bump version of the application and deploy it to the live environment, when there's a new commit to \emph{main} branch
   \end{itemize}
   
  \subsubsection{Deployment}
  A live environment is available for the FDS application. The FDS is deployed and run as a Docker container in \emph{Heroku}\footnote{\emph{Heroku} is a platform as a service (PaaS) that provides a platform for developer to build and run application in the cloud. Homepage: \url{https://heroku.com}.}. The deployment process will be executed via the CI/CD action on every commit to the \emph{main} branch. 

 \subsection{RabbitMQ}
 A RabbitMQ instance is required as one of the integral parts of the system. Fortunately, RabbitMQ provided an official Docker image for it and running a RabbitMQ instance as a Docker container is as simple as running the following command in a shell terminal:

 \begin{lstlisting}[caption={Running a RabbitMQ instance with Docker (Shell)}]
 docker run -it --rm --name rabbitmq -p 5672:5672 -p 15672:15672 rabbitmq:3.10-management 
 \end{lstlisting}

 The command listed above will locally run the RabbitMQ instance on port \emph{5672} and the RabbitMQ management UI on port \emph{15672}. On the live environment, a service called \emph{CloudAMQP}\footnote{\emph{CloudAMQP} is a service provider (RabbitMQ as a Service) that offers RabbitMQ clusters setup and management.} is used to help in setting up a RabbitMQ instance in the cloud, so that it can be accessed by other components via its public URI easily.

 \subsection{Redis}
 Redis is used in the system as a caching memory and a temporary data store. Redis also provided an official Docker image. To run a Redis instance as a Docker container, the following command should be run in a shell terminal:

 \begin{lstlisting}[caption={Running a Redis instance with Docker (Shell)}]
 docker run -d --name redis-stack-server -p 6379:6379 redis/redis-stack-server:latest
 \end{lstlisting}

 The command listed above will locally run the Redis instance in port \emph{6379}. Redis is not available on the live environment. On the live environment, no caching is available and an in memory data store (a basic JavaScript class) is used to replace Redis.

 \subsection{MongoDB}
 MongoDB is the database of choice for the system. MongoDB has an official Docker image and the following command should be run in a shell terminal to set up a MongoDB instance locally:

 \begin{lstlisting}[caption={Running a MongoDB instance with Docker (Shell)}]
 docker run --name mongodb -d -p 27017:27017 mongo 
 \end{lstlisting}

 By running the command listed above, a MongoDB instance will be run locally on port \emph{27017}. On the live environment, a service called \emph{MongoDB Atlas}\footnote{\emph{MongoDB Atlas} is a service that provides a cloud-hosted MongoDB instances. Homepage: \url{https://www.mongodb.com/atlas}.} will be used to help host a MongoDB instance in the cloud, making it more accessible by other components of the system.

 \subsection{Docker Compose}
 \textbf{TODO. Implementation not done}


\section{Object Models}
  \label{impl_model}

In this chapter, the specific implementation of the object models described in \autoref{subsection:model} will be discussed. 

  \subsection{Validation Rule}
    \label{impl_model:rule}

    The subsection describes the implementation of the \verb;ValidationRule; object model as a TypeScript interface.

    \subsubsection{TypeScript interface}
      As the implementation of the model defined in \autoref{fig:uml_validation_rule}, a TypeScript interface is created to provide a clear structure of a validation rule.

      \begin{lstlisting}[style=es6, caption={TypeScript interface of a validation rule (TypeScript)}]
export interface ValidationRule {
  retryStrategy?: RetryStrategy | null
  requestUrlParameter?: GenericObject
  requestHeader?: GenericObject
  skip: boolean
  requestBody?: GenericObject
  condition: Condition | BooleanCondition
  method: "GET" | "PUT" | "POST" 
  failScore: number
  endpoint: string
  priority: number
  name: string
}

type GenericObject = { [key: string]: any } // Dictionary
type Condition = {
  path: string
  operator: OperatorType
  type: ConditionType
  value: any
  failMessage: string
}

type BooleanCondition = {
  all: Condition[]
} | {
  any: Condition[]
}

type RetryStrategy = {
  limit: number
  statusCodes: number[] 
}
      \end{lstlisting}
 
    \subsubsection{Retry Strategy}
      \label{impl_model:rule__retry}

      The \verb;retryStrategy; attribute of the validation rule model is very specific to the implementation of the FDS. The FDS is using a library called \emph{Got}\footnote{\emph{Got} is a Node.JS library to make HTTP requests. GitHub repository: \url{https://github.com/sindresorhus/got}.} to make HTTP requests to the external endpoints. Got provides the functionality to retry a failed HTTP request out of the box\footnote{For more information on Got's retry options, please refer to \url{https://github.com/sindresorhus/got/blob/main/documentation/7-retry.md}.}.

      The \verb;retryStrategy; attribute of a validation rule is a subset of the retry options provided by Got's retry API, and will be passed to the Got instance when the HTTP request is made for the corresponding rule evaluation. 

  \subsection{Validation Result}
  
    A TypeScript interface is created as the implementation of the validation result structure listed on \autoref{fig:uml_validation_result}. The FDS is not responsible in storing validation results in a database. Therefore, a Prisma schema won't be created for validation results. 

    \begin{lstlisting}[style=es6, caption={TypeScript interface of a validation result (TypeScript)}]
export interface Validation<T> {
  validationId: string
  fraudScore: number
  totalChecks: number
  runnedChecks: number
  skippedChecks: string[]
  additionalInfo: ValidationAdditionalInfo<T>
  events: ValidationEvent[]
}

export type ValidationEventStatus = "NOT_STARTED" | "FAILED" | "PASSED" | "RUNNING"
export type ValidationEvent = {
  name: string
  status: ValidationEventStatus
  dateStarted: string | null
  dateEnded: string | null
  messages?: string[]
}

export type ValidationAdditionalInfo<T> = {
  startDate: string
  endDate?: string
  customerInformation?: T
}
    \end{lstlisting}
\section{Controller}

  This chapter describes the implementation of the features elicited on the previous chapters in details, specifically within the FDS component. All the HTTP routes of the FDS will be prefixed with \verb;/api/v1;.

  \subsection{Customer Validation on a Registration Event}

    An HTTP endpoint will be implemented to provide the possibility to schedule a validation process as soon as a new customer is registered. The endpoint should accept the customer information on the request body and return validation ID and additional information of the validation process as a response. Listed below is the code snippet of the HTTP controller of the endpoint to schedule a validation process:

    \begin{lstlisting}[style=es6, caption={HTTP controller of an endpoint to schedule a validation process (TypeScript)}]
// fds/src/routes/validation/validationController.ts

// Picks the `validationId` and `additionalInfo` attributes from `Validation` 
type ValidationSchedule = Pick<Validation, "validationId" | "additionalInfo"> 
     
@Route("validate")
@Tags("Validation")
export class ValidationController extends Controller {
  @SuccessResponse(201, "Validation started")
  @Response<ValidationErrorJSON>(422, "Validation Failed")
  @Response<WentWrong>(400, "Bad Request")
  @Post()
  public async validateCustomer(
   @Body() requestBody: Customer
  ): Promise<ValidationSchedule | WentWrong> {
    const result = await ValidationService.scheduleRulesetValidation(requestBody)
    const { data, error } = result

    if (error) {
      this.setStatus(400)
      return {
        message: error.message,
        details: error.details || "",
      }
    }

    return data
  }
} 
    \end{lstlisting}
    
    The HTTP controller is intentionally kept as simple as possible. The logic behind the process to schedule a validation is done by the \verb;ValidationService; and \verb;ValidationEngine; (discussed in \autoref{sub:process}). The \verb;ValidationService; is responsible in this particular case to get the lists of existing validation rules and runtime secrets (discussed in \autoref{sub:secrets}), then creating a new instance of \verb;ValidationEngine; as well as scheduling a new validation process. 

    \begin{lstlisting}[style=es6, caption={ValidationService schedule validation implementation (TypeScript)}]
// fds/src/routes/validation/validationService.ts

export class ValidationService {
  static async scheduleRulesetValidation(
    customer: Customer
  ): Promise<ApiResponse<ValidationSchedule>> {
    const { data: ruleset, error } = await RulesService.listRules()
    const secrets = await SecretsService.listSecrets()
    if (error) {
      return {
        data: null,
        error,
      }
    }

    const { validationId, additionalInfo } = await new ValidationEngine<Customer>()
      .setRuleset(ruleset)
      .setSecrets(secrets)
      .scheduleRulesetValidation(customer)

    return {
      data: {
        validationId,
        additionalInfo,
      },
      error: null,
    }
  }
}
    \end{lstlisting}

  \subsection{Validation Process}
    \label{sub:process}

    \subsubsection{Validation Process Flow}
    
      A validation process is started by iterating through a list of validation rules, making an HTTP request to the external endpoint listed on each rule and evaluating its response in comparison to the conditions attached on the rule. If the HTTP response from the external matches the conditions of the rule, the rule evaluation will be considered as a passed evaluation, otherwise it is a failed evaluation. The result of each rule evaluation determines the value of the resulting fraud score. The resulting fraud score will be calculated as follows: 

      \begin{itemize}
        \item Initialize an empty list of fraud scores. The list will be filled later with float numbers ranging between 0 and 1
        \item Go through the list of validation rules and run evaluation
        \item If the evaluation passed, append \verb;0; to the list of fraud scores
        \item Otherwise, append the validation rule \verb;failScore; attribute's value to the list of fraud scores
        \item At the end of the iteration, the list size should equal to the amount of available\footnote{\emph{Not skipped.}} validation rules
        \item The resulting fraud score is the sum of the scores in the list, divided by the number of available validation rules
      \end{itemize}
      
      In summary, the flow of a validation process will be represented by the flow diagram below:

      \begin{figure}[!ht]
        \centering
        \includegraphics[width=0.8\textwidth]{diagrams/flow_validation_process.jpeg}
        \caption{Flow diagram of a validation process}
        \label{fig:flow_validation}
      \end{figure}
      
      The flow listed above will be executed by creating a \verb;ValidationEngine; instance and calling a public \verb;secheduleRulesetValidation; method. Before running a validation process, the list of validation rules as well as runtime secrets should be provided to the engine. The \verb;ValidationEngine; class uses method chaining\footnote{\emph{Method chaining} is a method to provide the possibility of invoking multiple method calls of an object without having to store an intermediary result in an additional variable.} as well as the \emph{Builder} design pattern discussed in \autocite[pp. 97-106]{gamma-1995} for its construction, to make the public API of a validation engine as simple as possible.

      \begin{lstlisting}[style=es6, caption={ValidationEngine class and the scheduleRulesetValidation method (TypeScript)}]
// fds/src/engine/valdiationEngine.ts
export class ValidationEngine<T> {
  private secrets: GenericObject = {}
  private ruleset: ValidationRule[] = []

  setSecrets(secrets: GenericObject) {
    this.secrets = secrets
    return this
  }

  setRuleset(ruleset: ValidationRule[]) {
    this.ruleset = [...ruleset.sort((a, b) => b.priority - a.priority)]
    return this
  }

  async scheduleRulesetValidation(data: T): Promise<Validation<T>> {
    if (this.ruleset.length === 0) {
      throw new Error("Ruleset is not set")
    }

    await this.constructValidationObject(data) // Construct a validation result
    this.validateRuleset(data)

    return this.validationResult
  }

  async validateRuleset(data: T): Promise<Validation<T>> {
    if (this.ruleset.length === 0) {
      throw new Error("Ruleset is not set")
    }

    if (!this.validation) {
      await this.constructValidationObject(data)
    }

    for await (const rule of this.ruleset.filter(({ skip }) => !skip)) {
      const evaluationResult = await this.evaluateRule(rule, data)
      await this.reviewEvaluationResult(evaluationResult, rule)
    }

    await this.afterValidation()

    return this.validationResult
  }
}
      \end{lstlisting}

    \subsubsection{Making an HTTP Request to External Endpoints}

      A dedicated class (\verb;Agent;) is created to make an HTTP request to the external endpoint. The class will provide a layer of abstraction on top of the Got library that is being used to actually make the HTTP requests. 
      
      The class will also help in setting the request body, request header as well as to change the variables on the \verb;endpoint; attribute with its corresponding values. The class follows the \emph{Singleton} design pattern described in \autocite[pp. 127-134]{gamma-1995}, as there might only one instance needed for the whole application. The class is also implemented using the dependency injection in mind, for an easier access to the underlying library during the testing phase. 

      \begin{lstlisting}[style=es6, caption={Agent class, to make HTTP requests to the external endpoints (TypeScript)}]
// src/fds/engine/request/agent.ts
export class Agent {
  private static context: Context // Dependency injection

  private static get client() {
    return Agent.context.client
  }

  static setClient(context: Context) {
    this.context = context
  }

  static async fireRequest<DataType, ResponseType = string>(
    rule: ValidationRule,
    data: DataType,
  ): Promise<FireRequestReturnType> {
    const { method, retryStrategy } = rule

    try {
      const response = await this.client<ResponseType>(this.getUrl(rule, data), {
        method: method as Method,
        retry: retryStrategy || {},
        headers: this.getHeader(rule, data),
        json: this.getJSONBody(rule, data),
      })

      const { statusCode, statusMessage, body, rawBody, retryCount } = response

      return {
        error: null,
        data: {
          statusCode,
          statusMessage,
          rawBody,
          retryCount,
          body: this.parseResponseBody(body),
        },
      }
    } catch (e) {
      return {
        error: {
          message: e,
        },
        data: null,
      }
    }
  }
}
      \end{lstlisting}

      The attributes returned by the request agent is specifically chosen, as returning the whole response object might not be beneficial. 
      
      As there might be dynamic values listed on the \verb;requestBody, requestHeader; or \verb;requestUrlParameter; attributes of a validation rule\footnote{A dynamic value is marked by providing a JSONPath expression as the value in the key-value pairs of the corresponding attribute.}, a helper function to query the associated attribute by evaluating the JSONPath expression provided is implemented within the \verb;Agent; class. 

      \begin{lstlisting}[style=es6, caption={Accessing data on runtime (TypeScript)}]
// fds/src/engine/request/agent.ts
import jp from "jsonpath"

export class ValidationEngine<T> {
  private static accessDataFromPath([key, value]: [string, any], data: any) {
    if (typeof value !== "string") {
      return {
        [key]: value,
      }
    }

    const SEPARATOR = " "
    const splitted = value.split(SEPARATOR)
    return {
      [key]: splitted
        .map((chunk) => {
          try {
            const dataFromPath = jp.query(data, chunk)[0]
            return dataFromPath ?? chunk
          } catch {
            return chunk
          }
        })
        .join(SEPARATOR),
    }
  }
}
      \end{lstlisting}
      
    \subsubsection{Operators}
      - [ ] What
      - [ ] Flyweight
      - [ ] Example (number)?

      \verb;Operators; are special classes that define the operation of a certain condition during a validation process. Each \verb;Operator; is grouped by its type and has two main properties \verb;identifier;, and \verb;operateFunction;
      
      The \verb;identifier; property of an operator refers to the operator's name, unique on its group of type. The \verb;identifier; attribute of an operator will be passed into the \verb;operator; attribute of a condition to describe the specific operator to be used in evaluating the particular condition. The \verb;operateFunction; attribute of an operator is a function that accepts two arguments, and returns a boolean value that indicates whether the operation is successful. An additional validation process is also implemented using the \verb;validateFunction; property to make sure that the value being passed into the \verb;operate; function of an operator is valid.
      
      \begin{lstlisting}[style=es6, caption={The Operator class (TypeScript)}]
// fds/src/engine/operators/operator.ts
type OperateFunction<T, V = T> = (value: T, receivedValue: V) => boolean
export class Operator<T = any, V = T> {
  private identifier: string
  private operateFunction: OperateFunction<T, V>
  protected validateFunction: (value: any) => boolean

  constructor(identifier: IdentifierType, operateFunction: OperateFunction<T, V>) {
    this.identifier = identifier
    this.operateFunction = operateFunction
  }

  validate(value: any): boolean {
    return this.validateFunction(value)
  }

  setValidateFunction(validateFunction: (value: any) => boolean) {
    this.validateFunction = validateFunction
    return this // Method chaining
  }

  operate(value: T, receivedValue: V): boolean {
    const isValid = this.validate(value)
    if (!isValid) {
      return this.handleInvalidValue(value)
    }

    return this.operateFunction(value, receivedValue)
  }
}
      \end{lstlisting}
      
      \begin{lstlisting}[style=es6, caption={NumberOperator Example (TypeScript)}]
// fds/src/engine/operators/numberOperators
export class NumberOperator extends Operator<number, NumberOperatorIds> {
  const validateFunction = (value) =>
    typeof value === "number" &&
    !isNaN(parseFloat(`${value}`))
}

export const numberOperators: Record<NumberOperatorIds, NumberOperator> = {
  eq: new NumberOperator("eq", (a, b) => b === a),
  gt: new NumberOperator("gt", (a, b) => b > a),
  gte: new NumberOperator("gte", (a, b) => b >= a),
  lt: new NumberOperator("lt", (a, b) => b < a),
  lte: new NumberOperator("lte", (a, b) => b <= a),
} 
      \end{lstlisting}

      The \emph{Flyweight} design pattern mentioned in \autocite[pp. 195-206]{gamma-1995} is used here, by instantiating all the available operators beforehand, and using the instantiated object during a condition evaluation. For an even easier access to the operators, a \emph{flyweight factory} is also created. The \verb;OperatorFactory; will return the appropriate operator to be used based on the type and identifier passed. If the combination of type and identifier of an operator doesn't point into a specific operator, a \verb;NullishOperator; will be returned, which always return \verb;false; as its operation result. 
      
      \begin{lstlisting}[style=es6, caption={Operator factory (TypeScript)}]
// fds/src/engine/operators/operatorFactory.ts
export class OperatorFactory {
  static readonly nullishOperator = new Operator("null", () => false)
  private static readonly operatorMap = {
    (*@\textcolor{black}{string}@*): stringOperators,
    (*@\textcolor{black}{number}@*): numberOperators,
    array: arrayOperators,
    (*@\textcolor{black}{boolean}@*): booleanOperators,
  }

  static getOperator(type: ConditionType, operatorId: OperatorType): Operator {
    const operatorGroup = this.operatorMap[type]
    if (!operatorGroup) {
      return this.nullishOperator
    }

    const operator = operatorGroup[operatorId]
    if (!operator) {
      return this.nullishOperator
    }

    return operator as Operator
  }
}
      \end{lstlisting}

    \subsubsection{Evaluating a Rule} 
      - [ ] Evalutator
      - [ ] EvaluatorFactory, Abstract Factory pattern
      - [ ] Evaluating a rule in VDEngine
      
      To evaluate a certain validation rule, the \verb;Evaluator; class is created. 

      \begin{lstlisting}[style=es6, caption={Rule evaluation in ValidationEngine (TypeScript)}]
// fds/src/engine/validationEngine.ts
export class ValidationEngine<T> {
  private async evaluateRule(
    rule: ValidationRule,
    data: T,
  ): Promise<EvaluationResult> {
    const { endpoint, retryStrategy, condition, name } = rule

    const validationEvent = this.validation.events.find(
      (event) => event.name === name,
    )
    
    if (validationEvent) {
      validationEvent.dateStarted = new Date().toISOString()
      validationEvent.status = "RUNNING"
      await this.pushToDatastore()
    }

    const { error, data: responseData } =
      await Agent.fireRequest(rule, {
        customer: data,
        secrets: this.secrets,
      })

    if (error) {
      return {
        messages: [
          `${(*@\textcolor{black}{endpoint}@*)} is not accessible.${
            (*@\textcolor{black}{retryStrategy}@*)
              ? ` (*@\textcolor{forestgreen}{Retries done:}@*) ${retryStrategy.limit}`
              : ""
          }`,
          error.message,
        ],
        pass: false,
      }
    }

    const evaluator = EvaluatorFactory.getEvaluator(condition)
    return evaluator.evaluate({
      response: responseData,
      customer: data,
      secrets: this.secrets,
    })
  }
}
      \end{lstlisting}

  \subsection{Notification on Suspicious Cases}

  \subsection{Validation Rules Management}

  \subsection{Validation Real Time Progress}

  \subsection{Runtime Secrets}
    \label{sub:secrets} 

  \subsection{Error Handling}


\section{View}

  This chapter discusses the implementation of the features on the user interface described in \autoref{sub:design_view}.

  \subsection{Navigation}
  
    The user interface contains functionalities for several use cases, merged into a single web application. In a web application, routing plays a key role in determining what to show the user based on the URL address entered on the browser. Specifically in this case, routing can help in categorizing the current context of the application that consists of several views for different use cases. The following routes are implemented in the UI:

    \begin{itemize}
     \item Home page (path: \verb;/;)
     \item Create a new rule (path: \verb;/rules/new;)
     \item Edit a rule (path: \verb;/rules/<ruleName>;)\footnote{\emph{<ruleName>} refers to a dynamic value of a rule's name. Please take a look into \url{https://router.vuejs.org/guide/essentials/dynamic-matching.html} for more information.}
     \item List of rules (path: \verb;/rules;)
     \item Create a new validation (path: \verb;/validations/create;)
     \item See validation progress for specific validation ID (path: \verb;/validations/:validationId;)
     \item List of validations (path: \verb;/validations;)
    \end{itemize}
    
    To help the user in navigating the UI, a header is created and displayed on every page of the application. The header includes three buttons (\textsc{Home, Rules} and \textsc{Validations}), that links the user to the corresponding view of the application. 

    \begin{figure}[!ht]
     \includegraphics[width=\textwidth]{images/ss_navigation.jpeg}
     \caption{Screenshot of the header to navigate the UI}
    \end{figure}

  \subsection{Rule Management Form}
  
    The rule management form is a reusable form, can be used to both create a new and edit an existing validation rule. The rule management form displays all the attributes of a \verb;ValidationRule; model as a form field. 

    \begin{figure}[!ht]
      \includegraphics[width=\textwidth]{images/ss_sample_filled.jpeg}
      \caption{Screenshot of the rule management form}
    \end{figure}

    \subsubsection{Conditions Section}

      The \textsc{Conditions} section is a component, used to add one more condition to a validation rule. The \textsc{Conditions} section renders the list of conditions provided in a card, containing form fields for each attribute of the particular condition. Each card represents a single form, which also contains validation on its fields. 
      
      As mentioned before, the available values for the \verb;operator; attribute of a condition depends on its \verb;type; attribute. To prevent an invalid condition being sent to the FDS, the \textsc{Operator} field is a select field, and its options are defined by the current value inputted on the \textsc{Type} field. The intention of the restriction is to make sure that the \verb;operator; chosen is always valid to the corresponding \verb;type; of the condition.

      \begin{lstlisting}[style=es6, caption={Function to get list of available operators based on a condition's type attribute (TypeScript)}]
const getAvailableOperators = (type: ConditionType) => {
  switch (type) {
    case "string":
      return [
        { label: "Equals", value: "eq" },
        { label: "Starts with", value: "starts" },
        { label: "Includes", value: "incl" },
        { label: "Ends with", value: "ends" },
      ]
    case "number":
      return [
        { label: "Greater than", value: "gt" },
        { label: "Greater than equals", value: "gte" },
        { label: "Lesser than", value: "lt" },
        { label: "Leser than equals", value: "lte" },
        { label: "Equals", value: "eq" },
      ]
    case "array":
      return [
        { label: "Includes", value: "incl" },
        { label: "Excludes", value: "excl" },
        { label: "Number of items equals", value: "len" },
        { label: "Is empty", value: "empty" },
      ]
    case "boolean":
      return [{ label: "Equals", value: "eq" }]
    default:
      return []
  }
}
      \end{lstlisting}

      \begin{figure}[!ht]
        \centering
        \includegraphics[width=0.8\textwidth]{images/ss_condition_op.jpeg}
        \caption{Screenshot of the "Operator" select field options, based on the current "Type" value}
      \end{figure}
      
      If more than one condition is provided, a radio field is also rendered, so that the user can choose one of the provided modifier for the list of conditions (either \textbf{\emph{ALL}} or \textbf{\emph{ANY}}). 

      A form validation is also implemented in the \textsc{Conditions} section. The validation not only make sure that all the required fields are filled, but also the value of the fields itself. For example, the validation will display an error message if the \textsc{Type} field is set to "Number", but the input value of the \textsc{Value} field is not a valid number. 

    \subsubsection{Autocomplete Input}

    A JSONPath expression is a valid value for some attributes of a \verb;ValidationRule;, and it also might be needed to access the current runtime information during a validation process, such as the HTTP response from the external endpoint, runtime secrets and customer information. Unfortunately, it might be difficult to memorize the expressions needed to access certain values, and it might also confuse the user. 

    \begin{figure}[!ht]
      \centering
      \includegraphics[width=0.5\textwidth]{images/ss_autocomplete.jpeg}
      \caption{Screenshot of the autocomplete input usage}
    \end{figure}

    To solve this problem, an autocomplete input field is provided in certain fields where a JSONPath expression is used. User can then display a list of possible JSONPath expressions by prefixing an input field with "\$", and choosing one of the expressions listed. User can also extend the expression chosen. The autocomplete input is used in the following form fields:

    \begin{itemize}
      \item \textsc{Path} field on \textsc{Conditions} section
      \item \textsc{Value} field on \textsc{Conditions} section
      \item Value field of a dynamic input\footnote{Autocompletion on \emph{\$.response} is not available here.}
    \end{itemize}

  \subsection{Validation Form}

    The validation form is a simple form similar to a customer registration form, specifically to mimic a new customer registration and to run a validation process directly after a new registration. Form validation is also implemented in the validation form to make sure that the customer information sent to the FDS is complete. A set of sample customers with specific characteristic is created to provide the functionality of pre-filling the validation form. 

    \begin{figure}[!ht]
      \includegraphics[width=\textwidth]{images/ss_customer_form.jpeg}
      \caption{Screenshot of the validation form}
    \end{figure}

    \begin{lstlisting}[style=es6, caption={Prefilling form values with a sample customer data (TypeScript)}]
const applySampleCustomer = (sampleCustomer: Customer) => {
  Object.assign(formValues, {
    ...sampleCustomer,
  })
}
    \end{lstlisting}

    When the user filled the fields properly and clicked the \textsc{Validate customer} button, the UI sends an HTTP POST request to the FDS with the customer data as the payload to schedule a new validation process. As the FDS returns an ID of the validation process, the UI will then redirect the user to the validation progress page, showing the validation progress in real time. 

    \begin{lstlisting}[style=es6, caption={ (TypeScript)}]
const { data } = await createNewValidation(customer)
if (data.validationId) {
  router.push(`/validations/${data.validationId}`)
}
    \end{lstlisting}

  \subsection{Validation Progress}
  
    
  
  \subsection{Rule List and Validation List}

Optionals:
\begin{itemize}
  \item Architecture -> (can be in controller)
  \item Techs 
\end{itemize}
\chapter{Test}

  \begin{itemize}
   \item Setup with vitest on UI and FDS
  \end{itemize}

  \section{Unit Test}

  Unit tests play an important role during the development process, as it make sure that each component of the system works perfectly and fulfills its function in isolation.

  \subsection{FDS}

    The main component of the FDS, the \verb;ValidationEngine; consists of many components. To make sure that even the slightest change to the codebase won't break the whole system, each component is tested in isolation to ensure its function. Dependency injection pattern is also used here as a way to mock underlying third party libraries used by the component. 

    \begin{lstlisting}[style=es6, caption={Dependency injection usage in a unit test within FDS project (TypeScript)}]
describe("Agent", () => {
  const mockContext: MockContext = createMockContext()
  Agent.setClient(mockContext)

  afterEach(() => mockReset(mockContext))

  it("fires an HTTP request to the correct endpoint", async () => {
    await Agent.fireRequest(sampleRule, {})

    expect(mockContext.client).toBeCalledWith(sampleRule.endpoint, expect.anything())
  })
})
\end{lstlisting}

    \newpage
    \begin{lstlisting}[style=es6, caption={Example unit test of the condition evaluator (TypeScript)}]
describe("Condition Evaluator", () => {
  const condition: Condition = {
    path: "$.statusCode",
    operator: "eq",
    type: "number",
    value: 200,
    failMessage: "Status code doesn't equal to 200",
  }
  const evaluator = new ConditionEvaluator(condition)

  it("returns the correct result for a valid data", () => {
    const { pass, messages } = evaluator.evaluate({
      statusCode: 200,
    })

    expect(pass).toBeTruthy()
    expect(messages).toEqual([])
  })
})
\end{lstlisting}

  \subsection{UI}

    Testing a client-side web application might not be as straightforward in comparison to testing a server-side application. What's being tested in a client-side application is actually the behavior of each component and how it interacts to the user input. 
    
    Unit testing on the UI is done by isolating the smallest component and making sure it is behaving properly according to its specification. An additional library is needed to simulate the browser runtime-environment during testing. The library \emph{Vue testing library}\footnote{\emph{Testing library} is a family of packages for several front-end frameworks built to help test UI components. Homepage: \url{https://testing-library.com/}} is chosen as it provides several APIs to test the behavior of a UI component by resembling on what the user actually sees in the browser. 

    \begin{lstlisting}[style=es6, caption={Example unit test of a UI component (TypeScript)}]
describe("Key-Value input", () => {
  const renderComponent = (options: RenderOptions) => render(KeyValueInput, options)
  
  it("renders a new key-value input fields when `Add` is clicked", async () => {
    const { getByRole, queryAllByTestId } = renderComponent({})
    const addButton = getByRole("button", {
      name: "Add" 
    })

    expect(queryAllByTestId("key-value-field").length).toBe(0)
    await fireEvent.click(addButton)

    expect(queryAllByTestId("key-value-field").length).toBe(1)
    await fireEvent.click(addButton)

    expect(queryAllByTestId("key-value-field").length).toBe(2)
  })
})
\end{lstlisting}

  \section{Integration Test}

  \begin{itemize}
   \item Integration on FDS -> Endpoint test
   \item Use superagent and on
   \item UI -> Check that usage of other components are ok, minimal mocking
  \end{itemize}
\section{Demonstration}

  In this chapter, the following procedure will be demonstrated as the result of this research project:

  \begin{itemize}
   \item Retrieve a list of existing validation rules
   \item Creating a runtime secret (API key) to be used in one of the validation rule 
   \item Create a new validation rule
   \item Retrieve a list of validation processes
   \item Schedule a new validation process
   \item Email a specified email address if the fraud score of the validation exceeds \verb;0.5;
  \end{itemize}

  For demonstration purposes, an AMQP consumer is implemented to email a specified email address if the fraud exceeds a certain number\footnote{The AMQP consumer for the demonstration purpose is available as an attachment to this thesis}. Furthermore, an external address verification API\footnote{The external address validation used for the demonstration is Lob address verification. Homepage: \url{https://www.lob.com/address-verification}.} will be used as the external service for this demonstration. Therefore, an API key for the corresponding API needs to be created and added to the runtime secret of the validation engine. 

  After running the application using the command \verb;docker-compose up;, the UI is accessible on \verb;http://localhost:3000; and the FDS is accessible on \verb;http://localhost:8000;. The URL \verb;http://localhost:3000/#/rules; can be visited to display the list of available validation rules in the database. A new runtime secret can be created from the rules list page by clicking the \textsc{Secrets} button and then \textsc{Create a new secret} button afterward. 

  \begin{figure}[!ht]
   \centering
   \includegraphics[width=0.6\textwidth]{images/create_secret.jpeg}
   \caption{Screenshot of a runtime secret creation}
  \end{figure}

  The lob API uses the \emph{Basic Auth} authentication scheme, which requires a client to send the \verb;Authorization; header that contains the \verb;"Basic "; prefix, followed by the base64 encoded value of the Lob API key. As the runtime secret is created, it is now accessible to the validation rule using the \verb;$.secrets.LOB_API_KEY; JSONPath expression. The creation of a new validation rule can be started by clicking on the \textsc{Add new rule} button. A rule management form is displayed, and the user can enter the values to each attribute of the validation rule. For the demonstration, the following validation rule is created: 

  \begin{itemize}
   \item Name: "Address Validation"
   \item Endpoint: \verb;https://api.lob.com/v1/intl_verifications;
   \item HTTP method: \verb;POST;
   \item Fail score: 0.5
   \item Request body: 
     \begin{itemize}
       \item \verb;recipient: "FDS";
       \item \verb;primary_line: $.customer.address.street;
       \item \verb;city: $.customer.address.city;
       \item \verb;state: $.customer.address.state;
       \item \verb;country: $.customer.address.country;
     \end{itemize}
   \item Request header: \verb;Authorization: Basic $.secrets.LOB_API_KEY;
   \item Conditions:
     \begin{itemize}
      \item Evaluate whether the status code equals to 200
      \item Evaluate whether the response body returns a \verb;valid_address; attribute and whether it equals to \verb;true;
     \end{itemize}
  \end{itemize}
  
  By visiting the URL \verb;http://localhost:3000/#/validations;, the list of ongoing and completed validation processes, saved in the data store of the FDS will be displayed. A new validation process can be created by clicking on the \textsc{Create new validation} button and filling in the validation form with a sample customer data, on which a validation process should be executed. As mentioned before, several sample customers with different attributes are created to provide an even easier testing process. For this demonstration, the user with the label \verb;Berlin-based customer (invalid address); will be chosen, as it will trigger a failed rule evaluation of the \verb;Address Validation; validation rule. 

  \begin{figure}[!ht]
   \centering
   \includegraphics[width=\textwidth]{images/ss_validation_list.jpeg}
   \caption{Screenshot of the validation list page}
  \end{figure}

  After filling the validation form with a sample customer data, a validation process will be scheduled by clicking on the \textsc{Validate customer} button. As a validation process is scheduled, the user will be redirected to a validation process page, on which the progress of the validation process will be displayed in real-time. 

  Upon the completion of the validation process, validation result should return a \verb;0.5; fraud score, as the sample customer's address is invalid. As mentioned earlier, a certain action should be done by the AMQP consumer, when there's a validation process that resulted in a fraud score of \verb;0.5;. An email should be sent to the email address, specified when running the AMQP consumer. 
  
  \begin{figure}[!ht]
   \centering
   \includegraphics[width=0.6\textwidth]{images/ss_email.jpeg}
   \caption{Screenshot of an email sent by AMQP consumer when a validation process completed with a fraud score exceeding 0.5}
  \end{figure}
\chapter{Conclusion}

  In this chapter, the results and findings from previous chapters are summarized and an outlook on future improvements of the system is discussed.

  \section{Summary}

    Fraud detection and prevention plays a crucial role in building a profitable business by reducing the risk of damage done by any fraudulent activity. Furthermore, for organizations that utilize the autonomous teams structure, establishing a robust system for collaboration is important, so that the teams can work collectively on the primary goal of the organization.

    This research project aims to introduce a system that enables independent teams to collaborate in a fraud detection process, by creating validation rules as their contribution. Each team could create validation rules to validate certain characteristics of a customer, based on their domain knowledge and views on how a fraudulent customer might be. All validation rules created by different teams will be evaluated by the system and as the result, the system outputs a probability of the corresponding customer being a fraud, by combining the result of every validation rule evaluation into a single numerical value, ranging from zero to one. 
    
    The system publishes each validation result into an exchange of a messaging system, that will broadcast the message into every message consumer bound to the exchange. Additional independent systems can be implemented as a message consumer to consume the message from the exchange and run actions accordingly. A UI is also provided as a  graphical interface to manage validation rules and display the progress of a validation process in real time. 
  
  \section{Outlook}

    The system built in this research project is far from perfect. Systems are living products that can and should be improved over time. One of the possible major improvement of the system is the application of any authentication on the system. By implementing some sort of authentication mechanism, the security aspect of the system can be hugely improved, and any unauthorized use can be prevented. 

    Furthermore, it might be sometimes useful to group the validation rules, and run the rule evaluations only for certain groups. By assigning each rule into a group, there can be multiple types of validation processes, which can be modified based on the needs and requirements of the particular use case. Last but not least, a solution to make sure that the data of the customer is safe and compliant with the current GDPR rules is definitely needed. 

\printbibliography[
  heading=bibintoc,
  title={Bibliography}
]

\newpage

\chapter{List of Abbreviations}

\begin{tabularx}{\linewidth}{p{0.2\textwidth} p{0.8\textwidth}}
  AMQP & Advanced Message Queue Protocol \\ 
  API & Application Programming Interface \\ 
  B.C. & Before Christ \\ 
  CI/CD & Continuous Integration / Continuous Delivery \\ 
  CRUD & Create, Read, Update and Delete \\
  DB & Database \\
  FDS & Fraud Detection \emph{Service} \\ 
  GDPR & General Data Protection Regulation \\
  HTTP & Hypertext Transfer Protocol \\ 
  HW & Hardware \\ 
  I/O & Input / Output \\
  ISO & International Organization for Standardization \\
  JSON & JavaScript Object Notation \\ 
  % MVC & Model-View-Controller \\ 
  NPM & Node Package Manager \\
  ORM & Object-Relational Mapping \\ 
  REST & Representational State Transfer \\ 
  SSE & Server-Sent Events \\ 
  SW & Software \\ 
  UI & User Interface \\ 
  URL & Uniform Resource Locator \\ 
  URI & Uniform Resource Identifier \\ 
\end{tabularx}
\newpage
\include{others/glossary}

\appendix
\pagenumbering{Roman}

\chapter{Appendix}


\section{Supplemental Figures}

\begin{figure}[!ht]
 \includegraphics[width=\textwidth]{diagrams/entity_customer.jpeg}
 \caption{UML diagram of the customer model}
 \label{fig:customer_uml}
\end{figure}

\section{Quell-Code}

\section{Tipps zum Schreiben Ihrer Abschlussarbeit}

\begin{itemize}
\item Achten Sie auf eine neutrale, fachliche Sprache. Keine \glqq{}Ich\grqq{}-Form.
\item Zitieren Sie zitierf\"ahige und -w\"urdige Quellen (z.B. wissenschaftliche Artikel und Fachb\"ucher; nach M\"oglichkeit keine Blogs und keinesfalls Wikipedia\footnote{Wikipedia selbst empfiehlt, von der Zitation von Wikipedia-Inhalten im akademischen Umfeld Abstand zu nehmen \autocite{wikipedia2019}.}). 
\item Zitieren Sie korrekt und homogen.
\item Verwenden Sie keine Fu{\ss}noten f\"ur die Literaturangaben.
\item Recherchieren Sie ausf\"uhrlich den Stand der Wissenschaft und Technik.
\item Achten Sie auf die Qualit\"at der Ausarbeitung (z.B. auf Rechtschreibung).
\item Informieren Sie sich ggf. vorab dar\"uber, wie man wissenschaftlich arbeitet bzw. schreibt:
\begin{itemize}
\item Mittels Fachliteratur\footnote{Z.B. \autocite{balzert2011}, \autocite{franck2013}}, oder
\item Beim Lernzentrum\footnote{Weitere Informationen zum Schreibcoaching finden sich hier: \url{https://www.htw-berlin.de/studium/lernzentrum/studierende/schreibcoaching/}; letzter Zugriff: 13 VI 19.}.
\end{itemize}
\item Nutzen Sie \LaTeX\footnote{Kein Support bei Installation, Nutzung und Anpassung allf\"alliger \LaTeX-Templates!}.
\end{itemize}



\newpage
% Letzte Seite
\thispagestyle{empty}      
\noindent

\newpage
\input{appendix/versicherung}

\end{document}

